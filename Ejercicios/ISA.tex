\section{Arquitectura de programación}

\subsection{}
	  \begin{small}
			\begin{verbatim}                                   
			Instruction     gap    gcc    gzip   mcf    perl   I. average
			load           44.7%   35.5%  31.8%  33.2%  41.6%   37%
			store          10.3%   13.2%  5.1%   4.3%   16.2%   10%
			add             7.7%   11.2%  16.8%  7.2%   5.5%    10%
			sub             1.7%   2.2%   5.1%   3.7%   2.5%    3%
			mul             1.4%   0.1%                         0%
			compare         2.8%   6.1%   6.6%   6.3%   3.8%    5%
			cond branch     9.3%   12.1%  11.0%  17.5%  10.9%   12%
			cond move       0.4%   0.6%   1.1%   0.1%   1.9%    1%
			jump            0.8%   0.7%   0.8%   0.7%   1.7%    1%
			call            1.6%   0.6%   0.4%   3.2%   1.1%    1%
			return          1.6%   0.6%   0.4%   3.2%   1.1%    1%
			shift           3.8%   1.1%   2.1%   1.1%   0.5%    2%
			and             4.3%   4.6%   9.4%   0.2%   1.2%    4%
			or              7.9%   8.5%   4.8%   17.6%  8.7%    9%
			xor             1.8%   2.1%   4.4%   1.5%   2.8%    3%
			other logical   0.1%   0.4%   0.1%   0.1%   0.3%    0%
			load FP                                             0%
			store FP                                            0%
			add FP                                              0%
			sub FP                                              0%
			mul FP                                              0%
			div FP                                              0%
			mov reg-reg FP                                      0%
			compare FP                                          0%
			cond mov FP                                         0%
			other FP                                            0%
			\end{verbatim}
			\end{small}
		
			Considerar el agregado de un nuevo modo de acceso a MIPS.
			El nuevo modo suma dos registros y un valor de offset de 11 bits con signo para obtener
			la dirección efectiva. Utilizar el porcentaje de instrucciones indicado en la tabla anterior.
			\vspace{3mm}
			El compilador pasa de las instrucciones: 
			\begin{verbatim}
			    add R1, R1, R2
			    lw  Rd, 100(R1) #(o store)
			\end{verbatim}
			\vspace{1.5mm}
			A utilizar:
			\begin{verbatim}
			    lw	Rd, 100(R1,R2)
			\end{verbatim}
			
			\begin{enumerate}[label=\alph*)]
			 \item Asumir que el nuevo modo de acceso es usado por el 10\% de los loads y stores. ¿Cual es el porcentaje de Ic nuevo comparado con la tasa original?
            \item Si el nuevo modo de direccionamiento aumenta en 5\% el tiempo de clock, ¿Cuál
			       computadora será más rápida y por cuanto?
			\end{enumerate}

\subsection {}
            Al diseñar un sistema de memoria, es importante contrastar la frecuencia de accesos de lectura y accesos de escritura, así como
            accesos para leer instrucciones y accesos para datos.

            Indicar: 
            \begin{itemize}
            \item Porcentaje de accesos a memoria que son accesos a datos.
            \item Porcentaje de accesos de datos que son lecturas  
            \item Porcentaje de accesos a memoria que son lecturas
            \end{itemize}

\subsection{}
            Considere dos implementaciones $M1$ y $M2$ del mismo \textit{ISA}. Estamos interesados en la performance de dos programas $P1$ y $P2$, que tienen 
            la siguiente mezcla de instrucciones: 
            
            \begin{center}
            \begin{tabular}{||c|c|c||}
            \hline
            Operación & $P1$ & $P2$ \\
            \hline
            Load/Store & $40\%$ & $50\%$ \\
            \hline
            ALU        & $50\%$ & $20\%$ \\
            \hline
            Branches   & $10\%$ & $30\%$ \\
            \hline
            \end{tabular}
            \end{center}

            El \textit{CPI} para cada computadora es:
            \begin{center}
            \begin{tabular}{||c|c|c||}
            \hline
            Operación & $M1$ & $M2$ \\
            \hline
            Load/Store & $2$ & $2$ \\
            \hline
            ALU        & $1$ & $2$ \\
            \hline
            Branches   & $3$ & $2$ \\
            \hline
            \end{tabular}
            \end{center}

            \begin{itemize}
            \item Asuma que la frecuencia del reloj de $M1$ es 2GHz. ¿Cuál debería ser la frecuencia del reloj de $M2$ para que ambas
            computadoras reporten el mismo Tiempo de ejecución para $P1$?

            \item Asuma que ambas computadoras tienen la misma frecuencia de reloj, y que  $P1$ Y $P2$ ejecutan la misma cantidad de instrucciones.
            ¿Qué computadora es mas rápida para un conjunto de trabajo que consiste de igual número de ejecuciones de $P1$ y $P2$?

            \item Determine un conjunto de trabajo tal que $M1$ y $M2$ tengan la misma performance cuando tienen la misma frecuencia de reloj

            \end{itemize}
\subsection{}
      Pasar el siguiente código C a MIPS.
      \begin{small}
          \begin{lstlisting}[style=6620C]
              /*
              Asumir que las seis variables (f,g,h,i,j,k) corresponden a seis registros 
              ($s0,$s1,$s2,$s3,$s4,$s5), y existe una variable $t2=4.
              */
               
              switch (k) {
                  case 0:
                           f = i + j; break; 
                  case 1:
                           f = g + h; break;
                  case 2:
                           f = g - h; break; 
                  case 3:
                           f = i - j; break; 
              }
          \end{lstlisting}
      \end{small}
      \textit{Pista: usar la variable k para indexar una tabla de saltos, y luego 
      saltar a dichos valores.
      Primero verificar que el valor k se corresponda con una de las opciones posibles
      (0 \(<=\) k \(<=\) 3 ). Si no, salir.}
     \vfill    

\subsection{}
     Codificar en Assembly MIPS y diagramar el stack de las siguientes funciones.
           \begin{small}
             \begin{lstlisting}[style=6620C]
             
             void proc(int i){
               int j;
               j = i+20; 
             } 

             int 
             main(int argc, char** argv)
             {
                 const int i=10; 
                 proc(i); 
                 return 0; 
             } 
           \end{lstlisting}
     \end{small}

\subsection{}
  Codificar en Assembly MIPS y diagramar el stack de la siguiente función.
		\begin{small}
			\begin{lstlisting}[style=6620C]
				unsigned int
				factorial (unsigned int n)
				{
				    if (n < 2)
				        return 1;
				    else 
				        return n*factorial(n-1);
				}
			\end{lstlisting}
		\end{small}

\subsection{}
  Codificar en C  y Assembly MIPS una función que reciba calcule la longitud de un C-string\\
  \begin{lstlisting}[style=6620C]
  size_t strlen(const char *s)
  \end{lstlisting}
