\section{Architecture Binary Interface}
Esta sección no pretende ser una descripción completa de la ABI presentada en \cite{abi}, sino un detalle de los puntos que se consideran confusos, más importantes o que varían respecto al documento mencionado.

Esta explicación se basa en \cite{abi} y en las observaciones realizadas sobre código assembly generado con GCC, con nivel de optimización 0 (−O0).

Por lo tanto, se recomienda primero leer la sección “Function Calling Sequence” de \cite{abi}, para luego aclarar y/o modificar con el contenido de este documento.

\subsection{Caller y Callee Saved Registers}
Los siguientes registros deben ser salvados por la función llamada si ésta los debe modificar:

\begin{itemize}
 \item ra, sp, fp (o s8), gp, s0...s7
 \item f20 a f30 (floating point)
\end{itemize}

El resto de los registros no se garantiza que su valor se preserve entre llamadas a funciones. Si la función caller necesita preservarlos, debe salvarlos ella en su stack frame.

\subsection{Stack Frame}
Cada función crea su Stack Frame siempre.

El Stack Frame se compone de areas de un tamaño múltiplo de 8 bytes, alineadas a 8 bytes.

Los registros en un area se almacenan de abajo hacia arriba, en orden según el número de registro.

Las areas son, en orden de aparición (de arriba hacia abajo en memoria):

\begin{itemize}
 \item General Register Save Area (obligatoria). Registros salvados:
 \begin{itemize}
  \item Siempre: fp, gp;
  \item Cuando la función es non-leaf: ra;
  \item Si es necesario: el resto de los general purpose callee-saved regs.
 \end{itemize}
\item Floating-Point Registers Save Area: f20 a f30 si es necesario (ver 3-15 de \cite{abi} para más detalle)
\item Local and Temporary Variables Area;
\item Argument Building Area.
\begin{itemize}
 \item Se crea cuando la función es non-leaf
 \item Al menos es de 16 bytes, aún cuando los argumentos del callee
requieren menos.
 \item Cuando los cuatro primeros argumentos son enteros o punteros,
se pasan en a0 a a3, y el resto se guarda en esta area a partir de
los primeros 16 bytes.
 \item Los argumentos pasados en a0 a a3 son almacenados en esta area
por el callee siempre.
\end{itemize}
\end{itemize}

\begin{table}[!h]
\begin{center}
\begin{tabular}{|c|}\hline
Saved Regs Area (*)\\\hline
Float Regs Area\\\hline
Local Area\\\hline
Arg Building Area
$\leq$ 16 bytes\\\hline
\end{tabular}
\caption{Layout genérico del stack frame. Las areas marcadas con (*) son obligatorias.}
\end{center}
\end{table}


Debugging: algunas reglas de creación de stack frame existen sólo para
garantizar que las herramientas de debugging (ej: gdb) puedan puedan
realizar un stack backtrace (es decir, algunas reglas pueden no seguirse y
el programa funcionará correctamente, pero no se garantiza un correcto
analisis post-mortem en caso de que el proceso termine de forma anormal).

\begin{table}[!h]
\begin{center}
\begin{tabular}{|c|}\hline
fp (\$30)\\\hline
gp (\$28)\\\hline
\end{tabular}
\caption{Ejemplo de Saved Regs Area mínima - función leaf.}
\end{center}
\end{table}


\begin{table}[!h]
\begin{center}
\begin{tabular}{|c|}\hline
alignment [4]\\\hline
ra (\$31)\\\hline
fp (\$30)\\\hline
gp (\$28)\\\hline
\end{tabular}
\caption{Ejemplo de Saved Regs Area mínima para una función non-leaf. También puede salvar el resto de los callee-saved registers.}
\end{center}
\end{table}
